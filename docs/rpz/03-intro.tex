\section*{ВВЕДЕНИЕ}
\addcontentsline{toc}{section}{ВВЕДЕНИЕ}

\par За последнии 10 лет рынок аудиостриминга значительно вырос и претерпел сильные изменения\cite{robeco}.
Всё больше людей пользуются сервисами потокового прослушивания музыки, подкастов и аудиокниг.

\par Иследование \cite{scirp} показало, что число пользователей мобильных устройств превысило $2.5$ миллиарда человек.
Ежегодный прирост в $5-10\%$ пользователей способствет к стремительному увеличению популярности мобильных сервисов.
В результате возникает потребность в поддержке воспроизведения потокововых аудиоданных на мобильных устройствах.

\par Для реализации мобильного программного обеспечения необходимы программные комплексы, 
поддерживающие современные мобильные операционные системы и предоставляющие готовые программные инструменты для потокового воспроизведения аудиоданных.

\par На текущий момент количество программных комплексов с открытым исходным кодом для воспроизведения потоквого аудио,
поддерживающих различные протоколы передачи потоковых данных и частоты дескретизации, 
наложенение звуковых эффектов и варьирование качества звука,
для мобильных приложений на операционной системе Android значительно больше, чем для операционной системы iOS \cite{iOSOS}.
  
\par Целью работы является разработка программно-алгоритмического комплекса для воспроизведения потокового аудио в мобильном приложении на операционной системе iOS.
В рамках работы необходимо решить следующие задачи:

\begin{itemize}
    \item[---] ввести основные понятия предметной области;
    \item[---] рассмотреть существующие средства воспроизведения аудиоданных в операционной системе iOS;
    \item[---] провести анализ протоколов потоковой передачи данных;
    \item[---] рассмотреть форматы хранения аудиоданных;
    \item[---] спроектировать и реализовать программно-алгоритмический комплекс для воспроизведения потокового аудио в мобильном приложении на операционной системе iOS;
    \item[---] сравнить время и выделение оперативной памяти для обработки аудиоданных разработанным программным комплексом и существующими аналогами;
\end{itemize}
